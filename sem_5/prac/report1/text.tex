\tableofcontents
\clearpage

\section{Индивидуальное задание}

Ниже приведен текст индивидуального задания в соответствии с вариантом 10.

Разработать программу для хост-подсистемы и обработчики программного ядра, выполняющие следующие действия:
сетевой коммутатор на 128 портов. Сформировать в хост-подсистеме и передать в SPE таблицу коммутации из 254 ip адресов 195.19.32.1/24 (адреса 195.19.32.1 .. 195.19.32.254). Каждому адресу поставить в соответствие один из 128 интерфейсов (целые числа 0..127). Выполнить тестирование работы коммутатора, посылая из хост-подсистемы ip адреса и сравнивая полученный от GPC номер интерфейса с ожидаемым.

\section{Листинги программы}

В листинге \ref{lst:struct} представлен код файла common\_struct.h.

\lstinputlisting[label=lst:struct]{common\_struct.h}

В листинге \ref{lst:host} представлен код файла host\_main.cpp.

\lstinputlisting[label=lst:host]{host\_main.cpp}

В листинге \ref{lst:kernel} представлен код файла sw\_kernel\_main.cpp.

\lstinputlisting[label=lst:kernel]{sw\_kernel\_main.cpp}


\section{Результаты работы программы}

В листинге \ref{lst:log} представлен текст файла task.log.

\lstinputlisting[label=lst:log]{task.log}